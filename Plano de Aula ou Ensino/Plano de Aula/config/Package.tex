\usepackage[english,brazilian]{babel}
\usepackage[utf8]{inputenc}
\usepackage[T1]{fontenc}

%\usepackage{lmodern} %{lmodern}	% Usa a fonte Latin Modern			
\usepackage[top=20mm, bottom=20mm, left=30mm, right=30mm]{geometry}
\usepackage{framed}
\usepackage{booktabs} % Pacote para deixar tabelas mais bonitas.
\usepackage{color}	% Pacote de Cores
\usepackage[colorlinks=true,linkcolor=black,anchorcolor=black,citecolor=black,filecolor=black,menucolor=black,runcolor=black,urlcolor=black]{hyperref}%
\usepackage{graphicx}	% Pacote de imagens
\usepackage{microtype} 	% para melhorias de justificação
\usepackage{amsmath}
\usepackage[hang, small,labelfont=bf,up,textfont=it,up]{caption} % Custom captions under/above floats in tables or figures
\usepackage{indentfirst}% Indenta o primeiro parágrafo de cada seção.
\usepackage{enumitem} % Customized lists
\setlist[itemize]{noitemsep} % Make itemize lists more compact
\usepackage{array} %permite o uso de tabelas
\usepackage{longtable} %permite o uso de tabelas personalizadas
\usepackage{rotating} %permite rotacionar imagens
\usepackage{tikz} %permite o desenho de formas geométricas simples dadas suas características
\usepackage{ragged2e}%
\usepackage{csquotes}%
%\usepackage{setspace}%
\usepackage[style=apa]{biblatex} %coloque backref=true se quiser que fale em qual pagina foi citado
\addbibresource{referencias.bib} % Seus arquivos de bibliografia
\definecolor{shadecolor}{rgb}{0.9,0.9,0.9}
\graphicspath{ {./Images/} }
\usepackage{times}%\renewcommand\rmdefault{stb}