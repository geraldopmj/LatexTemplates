  \begin{snugshade}
  \section{Ementa}
  \end{snugshade}
  
  Escreva aqui a ementa do curso
  
  \begin{snugshade}
  \section{Objetivos} % a serem alcançados pelos alunos e não pelo professor. Podem ser divididos em gerais e específicos. 
  \end{snugshade}
  
    \subsection{Geral} % projeta resultado geral relativo a execução de conteúdos e procedimentos.

	Expor as diferentes metodologia utilizadas na análise de vestígios zooarqueológicos.\parencite{Albarella2001}
    
    \subsection{Específicos} % especificam resultados esperados observáveis (geralmente de 3 a 4).
    
%      \begin{itemize}
%      
%		\item teste
%     
%      \end{itemize}
  
  \begin{snugshade}
  \section{Conteúdos} % conteúdos programados para a aula organizados em tópicos (de 4 a 8).
  \end{snugshade}

Expor as metodologias:
      \begin{itemize}
      
		\item Quantitativas
		\item Qualitativas
      
      \end{itemize}

  \begin{snugshade}
  \section{Procedimentos metodológicos} % estratégias relevantes adotadas para alcançar os objetivos.
  \end{snugshade}

Exemplo: Conceitos a partir de aulas expositivas, com a utilização de slides em projetor multimídia na apresentação e desenvolvimento dos conteúdos, com promoção à participação dos alunos. Desenvolvimento de trabalhos e atividades, individuais e em grupo, como a apresentação de seminários, debates a partir de textos ou filmes e estudos de casos, objetivando a troca de conhecimento entre os alunos. Resolução de exercícios. Utilização do Google Meet para encontros semanais das aulas (síncronas). Uso do Google Sala de Aula (Classroom) para realização/envio de tarefas, disponibilização de materiais (artigos, textos, links, vídeos, OAs).

  \begin{snugshade}
  \section{Recursos didáticos} % quadro, giz, retro-projetor, filme, música, quadrinhos, etc.
  \end{snugshade}
  Exemplos:
    \begin{itemize}
	  \item apoio de recursos audiovisuais (data-show e filmes).
	  \item laboratório de informática.
	  \item laboratório multidisciplinar.
	  \item Google Sala de Aula (Classroom)
    \end{itemize}

  \begin{snugshade}
  \section{Avaliação} % pode ser realizada com diferentes propósitos (diagnóstica, formativa e somativa). Interessante explicitar a atividade avaliativa e os critérios de correção.
  \end{snugshade}
  Exemplo: A avaliação se dará por meio de provas e instrumentos diversificados, com participações de 70 por cento e 30 por cento, respectivamente, na composição da Média de Aproveitamento (MA). Se a média for maior 6,0 aprovado; Se a média for menor 6,0 Fazer prova de recuperação.

Havendo ausências nas provas, o aluno poderá requerer uma prova substitutiva correspondente, mediante solicitação formal ao setor de atendimento ao aluno, na Secretaria. A aprovação, independentemente das médias obtidas, está condicionada à frequência mínima de 75\%.