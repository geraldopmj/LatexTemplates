\documentclass[oneside,a4paper,12pt]{article}
%
%
%Esse template é uma alteração no modelo disponibilizado por Rodrigo Cavalcante (https://www.overleaf.com/articles/plano-de-aula/chhjxtdqjsqk).
%
%Separei o main.tex em vários arquivos para melhor localização e coloquei alguns pacotes para auxiliar na escrita e na bibliografia (biblatex-abnt). 
%
%No main.tex você define as informações do cabeçalho e tabela e no Texto.tex você escreve os objetivos, conteúdos, etc.
%
%É necessário colocar a bibliografia no referencias.bib e usar comandos do Biblatex. Os pacotes utilizados se encontra em Package dentro da pasta config.
%
%Caso mude para Plano de ENSINO mudar abaixo para Ensino na linha 17 e tirar o "%" na linha 36 deste arquivo para habilitar a seção Programação Semestral que contém tabela evidenciado o conteúdo de cada aula.
%
%
\usepackage[english,brazilian]{babel}
\usepackage[utf8]{inputenc}
\usepackage[T1]{fontenc}

%\usepackage{lmodern} %{lmodern}	% Usa a fonte Latin Modern			
\usepackage[top=20mm, bottom=20mm, left=30mm, right=30mm]{geometry}
\usepackage{framed}
\usepackage{booktabs} % Pacote para deixar tabelas mais bonitas.
\usepackage{color}	% Pacote de Cores
\usepackage[colorlinks=true,linkcolor=black,anchorcolor=black,citecolor=black,filecolor=black,menucolor=black,runcolor=black,urlcolor=black]{hyperref}%
\usepackage{graphicx}	% Pacote de imagens
\usepackage{microtype} 	% para melhorias de justificação
\usepackage{amsmath}
\usepackage[hang, small,labelfont=bf,up,textfont=it,up]{caption} % Custom captions under/above floats in tables or figures
\usepackage{indentfirst}% Indenta o primeiro parágrafo de cada seção.
\usepackage{enumitem} % Customized lists
\setlist[itemize]{noitemsep} % Make itemize lists more compact
\usepackage{array} %permite o uso de tabelas
\usepackage{longtable} %permite o uso de tabelas personalizadas
\usepackage{rotating} %permite rotacionar imagens
\usepackage{tikz} %permite o desenho de formas geométricas simples dadas suas características
\usepackage{ragged2e}%
\usepackage{csquotes}%
%\usepackage{setspace}%
\usepackage[style=apa]{biblatex} %coloque backref=true se quiser que fale em qual pagina foi citado
\addbibresource{referencias.bib} % Seus arquivos de bibliografia
\definecolor{shadecolor}{rgb}{0.9,0.9,0.9}
\graphicspath{ {./Images/} }
\usepackage{times}%\renewcommand\rmdefault{stb}
% Definições-----------------------------------------------------------
\newcommand{\plano}{Aula}%Ensino ou Aula
\newcommand{\universidade}{Universidade Federal de Minas Gerais (UFMG) \par Faculdade de Filosofia e Ciências Humanas (FAFICH) }
\newcommand{\departamento}{Departamento de Antropologia e Arqueologia}
\newcommand{\professores}{Joao Moreno}
\newcommand{\disciplina}{Zooarqueologia}
\newcommand{\tema}{Metodologias}
\newcommand{\prequisito}{Introdução à Arqueologia}
\newcommand{\data}{16/10/2021}
\newcommand{\tempodeaula}{45 minutos}
% ---------------------------------------------------------------------
\begin{document}
  \pagestyle{empty}
% Cabeçalho------------------------------------------------------------
	\begin{center}
	  \universidade
       \par
	  \departamento
	  \par
	  \vspace{10pt}
	  \LARGE \textbf{Plano de \plano}
	\end{center}
  \vspace{10pt}
%TABELA----------------------------------------------------------------
	\begin{tabular}{ |l|p{11.2cm}| }
	  \hline
	  \multicolumn{2}{|c|}{\textbf{Dados de Identificação}} \\
	  \hline
	  Professor(a):         &    \professores           \\
	  \hline
	  Disciplina:        &    \disciplina          \\
	  \hline
	  Tema:              &    \tema                \\
	  \hline
	  Pré-Requisitos:             &    \prequisito               \\
	  \hline
	  Data:              &    \data                \\
	  \hline
	  Duração da aula:   &    \tempodeaula         \\
	  \hline
	\end{tabular}

\vspace{3mm}
%Texto-----------------------------------------------------------------
  \begin{snugshade}
  \section{Ementa}
  \end{snugshade}
  
  Escreva aqui a ementa do curso
  
  \begin{snugshade}
  \section{Objetivos} % a serem alcançados pelos alunos e não pelo professor. Podem ser divididos em gerais e específicos. 
  \end{snugshade}
  
    \subsection{Geral} % projeta resultado geral relativo a execução de conteúdos e procedimentos.

	Expor as diferentes metodologia utilizadas na análise de vestígios zooarqueológicos.\parencite{Albarella2001}
    
    \subsection{Específicos} % especificam resultados esperados observáveis (geralmente de 3 a 4).
    
%      \begin{itemize}
%      
%		\item teste
%     
%      \end{itemize}
  
  \begin{snugshade}
  \section{Conteúdos} % conteúdos programados para a aula organizados em tópicos (de 4 a 8).
  \end{snugshade}

Expor as metodologias:
      \begin{itemize}
      
		\item Quantitativas
		\item Qualitativas
      
      \end{itemize}

  \begin{snugshade}
  \section{Procedimentos metodológicos} % estratégias relevantes adotadas para alcançar os objetivos.
  \end{snugshade}

Exemplo: Conceitos a partir de aulas expositivas, com a utilização de slides em projetor multimídia na apresentação e desenvolvimento dos conteúdos, com promoção à participação dos alunos. Desenvolvimento de trabalhos e atividades, individuais e em grupo, como a apresentação de seminários, debates a partir de textos ou filmes e estudos de casos, objetivando a troca de conhecimento entre os alunos. Resolução de exercícios. Utilização do Google Meet para encontros semanais das aulas (síncronas). Uso do Google Sala de Aula (Classroom) para realização/envio de tarefas, disponibilização de materiais (artigos, textos, links, vídeos, OAs).

  \begin{snugshade}
  \section{Recursos didáticos} % quadro, giz, retro-projetor, filme, música, quadrinhos, etc.
  \end{snugshade}
  Exemplos:
    \begin{itemize}
	  \item apoio de recursos audiovisuais (data-show e filmes).
	  \item laboratório de informática.
	  \item laboratório multidisciplinar.
	  \item Google Sala de Aula (Classroom)
    \end{itemize}

  \begin{snugshade}
  \section{Avaliação} % pode ser realizada com diferentes propósitos (diagnóstica, formativa e somativa). Interessante explicitar a atividade avaliativa e os critérios de correção.
  \end{snugshade}
  Exemplo: A avaliação se dará por meio de provas e instrumentos diversificados, com participações de 70 por cento e 30 por cento, respectivamente, na composição da Média de Aproveitamento (MA). Se a média for maior 6,0 aprovado; Se a média for menor 6,0 Fazer prova de recuperação.

Havendo ausências nas provas, o aluno poderá requerer uma prova substitutiva correspondente, mediante solicitação formal ao setor de atendimento ao aluno, na Secretaria. A aprovação, independentemente das médias obtidas, está condicionada à frequência mínima de 75\%.
% Programação Semestral------------------------------------------------
 
  \begin{snugshade}
  \section*{Programação Semestral} % 
  \end{snugshade}
\noindent	
	\begin{tabular}{|c|c|p{12cm}|}
	  \hline
	  \textbf{Aula}& \textbf{Data}  &  \multicolumn{1}{|c|}{\textbf{Conteúdo}} \\
	  \hline
	   01 & 01/01 & Apresentação da disciplina e diretrizes do curso.\\
	  \hline
	   02 & 01/01 &           \\
	  \hline
	   03 & 01/01 &           \\
	  \hline
	   04 & 01/01 &           \\
	  \hline
	   05 & 01/01 &           \\
	  \hline
	   06 & 01/01 &           \\
	  \hline
	   07 & 01/01 & Prova Bimestral \\
	  \hline
	   08 & 01/01 & Prova Substitutiva \\
	  \hline
	   09 & 01/01 &           \\
	  \hline
	   10 & 01/01 &           \\
	  \hline
	   11 & 01/01 &           \\
	  \hline
	   12 & 01/01 &           \\
	  \hline
	   13 & 01/01 & Encerramento do conteúdo e revisão de tópicos da disciplina \\
	  \hline
	   14 & 01/01 & Prova Final \\
	  \hline
	   15 & 01/01 & Prova de Recuperação \\
      \hline
	  
	\end{tabular}
\vspace{3mm}
% Referências bibliográficas-------------------------------------------
  \vspace{10pt}
\printbibliography[title={Bibliografia:}]%
\newpage
%Anexos----------------------------------------------------------------
  \begin{snugshade}
  \section*{Anexo I} % 
  \end{snugshade}
  
  \begin{figure}[h]
\includegraphics[width=15cm]{universe}
%\centering\caption{There's a map of concepts above.}
\end{figure}
%----------------------------------------------------------------------
\end{document}