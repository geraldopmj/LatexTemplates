%
%
%       NÃO MEXER AQUI
%     SOMENTE NO MAIN.TEX
%
%
%
\usepackage[english,brazilian]{babel}
\usepackage[utf8]{inputenc}%specifying to the engine how to process the symbols you're typing.
\usepackage[T1]{fontenc}%include the accented characters as individual glyphs.

%pacotes de fontes:
%\usepackage{lmodern} %{lmodern}	% Usa a fonte Latin Modern	
%\usepackage[defaultsans]{droidsans}
%\usepackage[default]{comfortaa}
%\usepackage{cmbright}
%\usepackage[default]{raleway}
%\usepackage{fetamont}
%\usepackage[default]{gillius}
%\usepackage[light,math]{iwona}
%\usepackage[thin]{roboto} 
\usepackage{times}% Usa a fonte Times new roman
%--------------------------------------------------------------
\usepackage[top=5mm, bottom=10mm, left=10mm, right=10mm]{geometry}%margens personalizadas
\usepackage{framed} % puts an ordinary frame box around the region,
\usepackage{booktabs} % Pacote para deixar tabelas mais bonitas.
\usepackage{color}	% Pacote de Cores
\definecolor{shadecolor}{rgb}{0.9,0.9,0.9}
\usepackage[colorlinks=true,linkcolor=black,anchorcolor=black,citecolor=black,filecolor=black,menucolor=black,runcolor=black,urlcolor=black]{hyperref}%sed to handle cross-referencing commands in LATEX to produce hypertext links
\usepackage{graphicx}	% Pacote de imagens
\graphicspath{ {./Images/} }
\usepackage{microtype} 	% para melhorias de justificação
\usepackage{amsmath} % ambiente matematico
\usepackage[hang, small,labelfont=bf,up,textfont=it,up]{caption} % Custom captions under/above floats in tables or figures
\usepackage{indentfirst}% Indenta o primeiro parágrafo de cada seção.
\usepackage{enumitem} % Customized lists
\setlist[itemize]{noitemsep} % Make itemize lists more compact
\usepackage{array} %permite o uso de tabelas
\usepackage{longtable} %permite o uso de tabelas em mais de uma pagina
\usepackage{rotating} %permite rotacionar imagens
\usepackage{tikz} %permite o desenho de formas geométricas simples dadas suas características
\usepackage{tabularx}
\usepackage{ragged2e}%set ragged text and are easily configurable to allow hyphenation
\usepackage{csquotes}%provides advanced facilities for inline and display quotations
%\usepackage{setspace}%
\usepackage[style=apa]{biblatex} %coloque backref=true se quiser que fale em qual pagina foi citado
\addbibresource{referencias.bib} % Seus arquivos de bibliografia
