\usepackage[utf8]{inputenc}
\usepackage[T1]{fontenc}
\usepackage{times} %{lmodern}	% Usa a fonte Latin Modern			
\usepackage[T1]{fontenc}% Selecao de codigos de fonte.
\usepackage[utf8]{inputenc}% Codificacao do documento (conversão automática dos acentos)
\usepackage{indentfirst}	% Indenta o primeiro parágrafo de cada seção.
\usepackage{color}		% Controle das cores
\usepackage{microtype} 	% para melhorias de justificação
\usepackage{amsmath}
\usepackage{lipsum}
\usepackage{}
\usepackage[a4paper,
%totalwidth=210mm,
%totalheight=297mm,
%margin=2cm,
headheight=1.3cm,
headsep=0.5cm,
%headwidth=17cm,
%textheight=24cm,
%textwidth=17cm,
footskip=1.5cm,
lmargin=2.5cm,
rmargin=2.5cm,
top=1cm,
bottom=2cm,
centering,
includefoot, 
includehead, 
]{geometry}
%\setlength{\hoffset}{-2cm}
%\setlength{\voffset}{-1cm}
%\setlength{\oddsidemargin}{1.5cm}
%\setlength{\evensidemargin}{1.5cm}
%\setlength{\topmargin}{-1cm}
%\setlength{\headheight}{1.2cm}
%\setlength{\headwidth}{17cm}
%\setlength{\headsep}{0.4cm}
%\setlength{\textheight}{24cm}
%\setlength{\textwidth}{17cm}
%\setlength{\footskip}{1.0cm}
%\setlength{\columnsep}{0.5cm}
%
%
\usepackage{graphicx}
\newcommand{\vcenteredinclude}[1]{\begingroup%
\setbox0=\hbox{\includegraphics{#1}}%
\parbox{\wd0}{\box0}\endgroup}%
%
\usepackage[export]{adjustbox}%
\usepackage{fancyhdr} %define o estilo do documento
%
\linespread{1.0} % Line spacing - Palatino needs more space between lines
%
%
\usepackage[hang, small,labelfont=bf,up,textfont=it,up]{caption} % Custom captions under/above floats in tables or figures
\usepackage{booktabs} % Horizontal rules in tables
\usepackage{blindtext}%
\usepackage{pgfornament}%
%\usepackage{pagecolor}%
%\usepackage{ebgaramond}
\usepackage{lettrine} % The lettrine is the first enlarged letter at the beginning of the text
%
\usepackage{enumitem} % Customized lists
\setlist[itemize]{noitemsep} % Make itemize lists more compact
\usepackage{array} %permite o uso de tabelas
\usepackage{longtable} %permite o uso de tabelas personalizadas
\usepackage{rotating} %permite rotacionar imagens
\usepackage{tikz} %permite o desenho de formas geométricas simples dadas suas características
\usepackage{ragged2e}%
\usepackage{csquotes}%
%
\usepackage{microtype}
\newcommand{\spacedtext}[1]{\ifXeTeX{\addfontfeature{LetterSpace=20}\scshape #1} \else \textls[170]{\scshape #1}\fi}%
\newcommand{\longs}{\ifXeTeX ſ\else s\fi}%

\usepackage{float}%
\usepackage{trimspaces}

\input GoudyIn.fd

\usepackage{titlesec} % Allows customization of titles
\renewcommand\thesection{\Roman{section}}% Roman numerals for the sections
\renewcommand\thesubsection{\roman{subsection}}% roman numerals for subsections
\titleformat{\section}[block]{\large\bfseries\scshape\centering}{\thesection.}{1em}{}% Change the look of the section titles
\titleformat{\subsection}[block]{\large}{\thesubsection.}{1em}{} % Change the look of the section titles
%
%\usepackage[colorlinks=true,linkcolor=black,anchorcolor=black,citecolor=black,filecolor=black,menucolor=black,runcolor=black,urlcolor=black]{hyperref}%
% ---
% Pacotes de citações
\usepackage[style=abnt]{biblatex} %coloque backref=true se quiser que fale em qual pagina foi citado
\addbibresource{referencias.bib} % Seus arquivos de bib
%
\usepackage{lipsum}
%----------------------------------------------------------
%\usepackage{eso-pic}
%
%\newcommand\AtPageUpperRight[1]{\AtPageUpperLeft{%
% \put(\LenToUnit{\paperwidth},\LenToUnit{0\paperheight})%{#1}%
 %}}%
%\newcommand\AtPageLowerRight[1]{\AtPageLowerLeft{%
% \put(\LenToUnit{\paperwidth},\LenToUnit{0\paperheight})%{#1}%
% }}%
%
%\AddToShipoutPictureBG{%
%   \AtPageUpperLeft{\put(0,-25){\pgfornament[width=1.75cm]{61}}}
%   \AtPageUpperRight{\put(-50,-25)%{\pgfornament[width=1.75cm,symmetry=v]{61}}}
%   \AtPageLowerLeft{\put(0,25)%{\pgfornament[width=1.75cm,symmetry=h]{61}}}
%   \AtPageLowerRight{\put(-50,25)%{\pgfornament[width=1.75cm,symmetry=c]{61}}}
%  }%

%-------------------------------------------------------------
%-------------------------------------------------------------
%INFOS PARA CABEÇALHO DA PAGINA TITULO
\fancypagestyle{fancytitle}{%
\fancyhf{}%
\renewcommand{\headrulewidth}{0pt}%~\\[-2ex]
\fancyhead[R]{\cabecalhodireitotitle}% Custom header text
\fancyhead[L]{\cabecalhoesquerdotitle}% Custom header text
\fancyfoot[C]{\thepage} % Custom footer text
}%
%----------------------------------------------------------
%INFOS PARA CABEÇALHO DO MEIO DO ARTIGO
\fancypagestyle{usuario}{ % All pages have headers and footers
\fancyhf{}%
\fancyhead[LO, RE]{\cabecalhodireito} % Custom header text
\fancyhead[RO, LE]{\cabecalhoesquerdo}% Custom header text
\fancyfoot[CE, CO]{\thepage} % Custom footer text
}%
%------------------------------------------------------

%-------------------------------------------------------------
%INFO PARA FORMATAÇÂO DO TÌTULO
\makeatletter% 
\def\@maketitle{%
\begin{center}%
~\\[5ex]
%\pgfornament[width=15cm,color = red!30!black]{71}~\\[5ex]%
%\includegraphics[width = 40mm]{image/logoeu.png}~\\[5ex]%
{\Huge \@title }~\\[0ex]%
%\pgfornament[width=15cm, ]{89}~\\[1ex]%
\hrulefill~\\[0ex]
{\@author}~\\[-5ex]%
{\@date}~\\[0ex]%
%\hline~\\[1ex]
\end{center}%
%\begin{abstract}
%\begin{center}
%\begin{singlespace}
%{\small \itshape Este e o resumo resumo resumo resumo resumo resumo resumo resumo resumo resumo resumo resumo resumo resumo resumo resumo resumo resumo resumo resumo resumo resumo resumo resumo resumo resumo  resumo resumo resumo resumo resumo resumo resumo resumo resumo.}
%\end{singlespace}
%\end{center}
%\end{abstract}~\\[0ex]
%\hfill \break
~\\[-10ex]
}%

\ifthenelse{\equal{\ABNTEXisarticle}{true}}{%
\renewcommand{\maketitlehookb}{}
}{}
\PassOptionsToPackage{brazil,french,english,spanish}{babel}

\makeatother%